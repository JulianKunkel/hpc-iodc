\documentclass[a4paper,10pt]{article}
\usepackage[utf8]{inputenc}
\usepackage{hyperref}

\title{HPC I/O in the Data Center}
\author{Julian Kunkel \and Jay Lofstead \and Jean-Thomas Acquaviva}

\makeatletter
\renewcommand{\@seccntformat}[1]{}
\makeatother

\setlength{\parindent}{0cm}
\setlength{\parskip}{0.5em}

\usepackage{url}

\begin{document}

\maketitle


\section{Abstract}
Managing scientific data at a large scale is challenging for scientists but also for the host data center.
The storage and file systems deployed within a data center are expected to meet users' requirements for data integrity and high performance across heterogeneous and concurrently running applications.

With new storage technologies and layers in the memory hierarchy, the picture is becoming murkier.
To effectively manage the data load within a data center, I/O experts must understand how users expect to use these new storage technologies and what services they should provide in order to enhance user productivity. We seek to ensure a systems-level perspective is included in these discussions.

The HPC-IODC workshop is a forum to present and discuss work that addresses the storage challenge from a unique perspective, moving the focus from the application-centric perspective to the perspective of data centers and operators.
In the workshop, we bring together I/O experts from data centers and application workflows to share current practices for scientific workflows, issues and obstacles for both hardware and the software stack, and R\&D to overcome these issues.
To focus on relevant aspects and streamline the discussion, a list of relevant topics is provided as the common structure of the talks.

The IODC workshop covers three tracks:
The research track accepts papers covering state-of-the-practice and research dedicated to storage in the data centre.
The expert talks track revolves around presentations such as a description of the operational aspects of a data centre and 
particular solutions for specific data centre workloads in production.
Lastly, in the student mentoring sessions, the students will be given 10 minutes to talk about what they are working on followed by 10-15 minutes of conversation with the community to help the students progress in their studies.

For further information regarding participation and the topics, see the web page.

\section{Keywords}
Data center, File systems, Storage, Performance, Architecture

\section{Workshop length}
Full day.

\section{Workshop webpage}
\url{https://hps.vi4io.org/events/2023/iodc}


\section{Targeted audience}
\begin{itemize}
\item I/O experts from data centers and industry.
\item Researchers/Engineers working on high-performance I/O for data centers.
\item Interested domain scientists and computer scientists interested in discussing I/O issues.
\item Vendors are also welcome, but their presentations must align with the same topics and not focus on commercial aspects.
\end{itemize}




\section{Estimated attendance}
40-50

\section{Early Acceptance}

We will contribute to the joint Springer LNCS post-conference proceedings but also invite researchers to publish extended articles in the Journal of High-Performance Storage (JHPS).


\section{Workshop format}
%The workshop content is structured in three parts, firstly, in the morning we organize a shared session with the I/O related WOPSSS workshop, which covers a keynote, best papers and community activities.
%Ideally, the morning session is held in a larger room and the two parallel sessions for WOPSSS and HPC-IODC in the afternoon in regular rooms.
%Both workshop will run separately in the afternoon on a more specialized and narrow topic.
The morning session offers a keynote, best papers, and community activities.
The afternoon session of HPC-IODC puts an emphasis on data center issues and covers other research paper presentations regarding state of the practice in the data center, discussion slots, and talks from I/O experts.
%We will live broadcast the presentations on Youtube using the channel of the  Virtual Institute for I/O.

Researchers and I/O experts can submit their proposals for a paper or talk according to a call for participation for papers and speakers.
We provide a common list of topics to be addressed in each I/O expert's talk on our webpage, such that the individual presentations are aligned.



\section{Previous Instances}

This workshop is held since the first time ISC extended it's program to include workshops (ISC 2015).
In 2015, the workshop was held with approximately 30 attendees, 10 invited presentations, and two discussion sessions.
In 2016, the workshop format was changed, in addition to the expert talks, we added a research section allowing submission of short scientific papers and a keynote talk.
The attendance to the workshop was similar to 2015, but finally we accepted four research papers.
In 2017, more than 50 people attended the workshop; we accepted 5 papers. % of 6
Despite the large attendance of previous instances, we were forced by the organizers to run on a half day.
That year we teamed up with WOPSSS (a workshop focusing on performance aspects in I/O) and organized a full storage together promoting the two events on a joint webpage: \url{https://isc-hpc-io.org}.
In 2018, we organized a shared morning session with WOPSSS (45+ people) and then split after lunch into two fully autonomous workshops.
From 2019 onward, WOPSSS moved to a different conference venue and, thus, HPC-IODC was again the only storage-centric workshop at ISC.
Attendance for 2019, 2020, 2022 and 2023 was between 40 and 50 people in the morning session, with a drop to around 30 in the afternoon session.
The number of research papers on these instances was just a handful but we always organized a program around expert talks and student research talks.

\section{Expected outcome}
1) Sparking impulses for the development of data center I/O.

2) The networking among the participants will be improved.

3) We contribute a summarizing workshop paper to ISC's post-conference workshop proceedings in LNCS.

\section{Details on the Call for Papers}

We accept short papers with up to 12 pages (excl. references) in LNCS format.
We cover data center research, state-of-the-practice papers, and reviewed expert talks based on a submitted up to 2-page abstract including a short bio.
Our targeted proceedings are ISC's post-conference workshop proceedings in Springers LNCS.
We will use Easy-chair for managing the proceedings and PC interaction.

Expected number of papers: We expect 5 submissions.
% With an acceptance rate of 50\%, we aim to fill at least half of the presentation slots with papers.

\paragraph{Tentative schedule:}
\begin{itemize}
  \item Announcement/CfP: As soon as workshop notifications are sent by ISC
  \item Submission deadline: 2024-03-01 AoE
  \item Author notification: 2024-03-29
\end{itemize}
Note that our call for speakers follows a similar schedule, but they'll have to submit only a talk abstract and a brief bio.

Prospect members of the program committee are:
\begin{itemize}
  \item Thomas Bönisch (HLRS)
  \item Suren Byna (Lawrence Berkeley National Laboratory) 
  \item Matthew Curry (Sandia National Laboratories)  
  \item Sandro Fiore (University of Trento)  
  \item Javier Garcia Blas (Carlos III University) 
  \item Stefano Gorini (Swiss National Supercomputing Centre) 
  \item Adrian Jackson (The University of Edinburgh) 
  \item Ivo Jimenez (University of California, Santa Cruz)
  \item George S. Markomanolis (AMD) 
  \item Sandra Mendez (Barcelona Supercomputing Center (BSC)) 
  \item Feiyi Wang (Oak Ridge National Laboratory) 
\end{itemize}

To increase the participation further, we firstly will actively invite representatives of different data centers and representatives of active workflow users/developers to give a talk and/or submit short papers regarding recent issues.
Secondly, we will announce the workshop on typical mailing lists and send the call for participation.


For further information, see:
\url{https://hps.vi4io.org/events/2024/iodc}.





\end{document}
