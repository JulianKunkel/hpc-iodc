\documentclass{superfri}

\usepackage[T1]{fontenc}
\usepackage[utf8]{inputenc}

\usepackage{multirow}
\usepackage{amsmath}
\usepackage{babel}
\usepackage[official]{eurosym}
\usepackage{graphicx}
\usepackage{multirow}
\usepackage[list=true,hypcap=true]{subcaption}
\usepackage{units}
\usepackage{xfrac}
\usepackage{color}

\usepackage{varioref}
\usepackage[hidelinks]{hyperref}
\usepackage[capitalise,noabbrev]{cleveref} % Version 0.18.10

\bibliographystyle{plain}
\numberwithin{equation}{section}


\begin{document}

% PAGE LIMIT: full paper 10-16 pages

\author{Julian M. Kunkel\footnote{\label{dkrz}Deutsches Klimarechenzentrum (DKRZ), Hamburg, Germany}, 
Jay Lofstead\footnote{Center for Computing Research, Sandia National Laboratories, Albuquerque, USA}, 
Colin McMurtrie\footnote{Swiss National Computing Center (CSCS), Lugano, Switzerland}, 
PLEASE ADD YOURSELF
} % \footnoteref{dkrz}

\title{Data Center Perspectives on HPC-IO}
\maketitle{}

\begin{abstract}

\noindent
\keywords{Parallel I/O, data center, file systems, managing data}
\end{abstract}

% -----------------------------------------------------------------------
\section{Introduction}
\label{sec:intro}

\cite{DOEExa13}

The DOE recognizes the importance of data management, listing it among the top 10 research challenges for Exascale \cite{top14}.
It says: “Affordability of data management, both procurement and operational, is a huge challenge.“

The complexity of the storage hierachy is increasing with the adoption of SSD and memory class storage technology.
Burden to data centers to make right choices...

Initiatives and consortia, 
Exascale10 workgroup \cite{brinkmann14},
Parallel Data Storage Workshop (PDSW), 
Big Data and Extreme-Scale Computing (BDEC), http://www.exascale.org/bdec/
HEC FSIO workshop \cite{bancroft2009} between 2005 and 2011, 
smaller workshops such as Clustor, 

Storage Networking Industry Association (SNIA) for enterprise
special tracks in conferences.

In 2013, an article was published that summarizes commercial data centers' dimensions \cite{data13}.
Facebook around 100\,PB of storage. Google and Microsoft hosted around 1 Million servers.

Global installed data storage capacity to hit 7,235 exabytes by 2017
\cite{http://dailyindependentnig.com/2013/10/global-installed-data-storage-capacity-to-hit-7235-exabytes-by-2017-report/}

Within 5 years Seagate shipped 1 billion HDDs, 700k every day 
\cite{Storage Solutions Guide - Seagate / http://www.seagate.com/files/www-content/product-content/_cross-product/en-us/docs/seagate-storage-and-application-guide-apac.pdf}
With state-of-the-art 8\,TB disks, this would already account for 5.5 Exabyte capacity by day.

Studies of I/O workloads. In \cite{luu2015multiplatform}, logs automatically captured with the Darshan tool are analyzed.

Studies: Software-Defined storage
\cite{http://datacore.com/sf-docs/default-source/whitepapers/english/the-state-of-sds-2015-survey.pdf}

cloud storage
\cite{http://www.ctera.com/enterprise-cloud-storage-survey-2015}
Siehe auch http://www.prnewswire.com/news-releases/2015-ctera-enterprise-cloud-storage-survey-highlights-data-governance-and-security-concerns-continuing-preference-for-private-and-virtual-private-clouds-300092189.html

Enterprise Storage Services Survey
\cite{http://www.idc.com/getdoc.jsp?containerId=254468}

Typically focus on the application, while we focus on the data center's perspectives.

Summary of the HPC-IODC workshop at ISC...
Contributions:
- contemporary survey of data center perspectives:
workloads, system view: utilization, monitoring, issues in production systems and needed R\&DE and potential solutions.

\section{Participating Data Centers}
\label{sec:centers}

\textit{Use this section as brief advertisement for all data centers and include a short description about key workloads.}
An overview of the characteristics of the data centers is given in \Cref{tbl:overviewCharacteristics}.
The I/O systems of the current supercomputers are characterized in \Cref{tbl:overviewIO}.


\paragraph{DKRZ}
Typical workloads.

\paragraph{CSCS}

\paragraph{HLRS}
Hornet


%%%%%%%%%%%%%%%%%%%%%%%%%%%%%%%%%%%%%%%%%%%%%%%%%%%%%%%%%%%%%%%%%%%%%%%

\begin{table}[bt]
\renewcommand{\arraystretch}{0.8}
\renewcommand{\tabcolsep}{0.1cm}
\begin{tabular}[c]{ll|r|r|r|r||r||r|r}
Center & Top    & \multicolumn{4}{c||}{Compute}      & \multicolumn{1}{c||}{Network}               & \multicolumn{2}{c}{Archive} \\
 & system & Peak perf. & Node cnt & Cores/Node & Total mem & Type                 & Tape slots & Peak perf. \\ \hline
 \hline

DKRZ & Mistral & 1.49\,PF   & 1500       & 24         & IB FDR                     & 65,000 & XY\,MiB/s \\ \hline
Sandia & Sky Brige & 615\,TF   & Intel Truscale \\ \hline
CSCS & Piz Daint & 7.79\,PF & Cray Aries \\ \hline
HLRS & Hornet & 3.79\,PF & 3944 & 24 & 493 TB & Cray Aries &   \\ \hline
% CSCS & 

\end{tabular}
\caption{Data center system characteristics\label{tbl:overviewCharacteristics}}
\end{table}

\begin{table}[bt]
\renewcommand{\arraystretch}{0.8}
\renewcommand{\tabcolsep}{0.1cm}
\begin{tabular}[c]{l|r|r|r|r|r|r|r}
Center & Type        & Capacity  & Peak perf.     & Server cnt & File systems & File count    & Avg file size    \\ \hline
DKRZ   & Lustre\,2.5 & 20\,PB    & 313\,GiB/s     & 60         & 1            &               &                  \\ \hline
HLRS   & Lustre  & 8.1\,PB       & 150\,GiB/s     & 112        & 7            & 3.5\,M        & 
% CSCS &

\end{tabular}
\caption{Data center storage characteristics\label{tbl:overviewIO}}
\end{table}

\section{Managing Storage}

Concepts.
DKRZ: Scratch, Work, Projects
DNE issues

HLRS: Work Space mechanism
• A directory in the project file system is created upon request
with a user defined name
• The directory is available for 30 days
• The directory life time can be extended 3 times by 30 days
• At the end of life, the directory with its content!!! is
automatically deleted
• There are tools for
– finding available workspaces
– Releasing workspaces
– Setting a reminder in calender tools
• Quota is enabled


\section{Monitoring}
\label{sec:monitoring}
Key issue.


\section{Production Issues}
\label{sec:issues}

Lustre:
- DNE issues
- High-level APIs such as NetCDF and HDF5 perform suboptimal, roughly 10\% of peak.
- Quota
- Full OSTs

\section{Conducted R\&D}
\label{sec:randd}

\section{Summary, Conclusions}
\label{sec:summary}


\ack{%
\noindent
Acknowledgment:}

\bibliography{literature}


\end{document}
