\documentclass{llncs}

\usepackage[T1]{fontenc}
\usepackage[utf8]{inputenc}

\usepackage[official]{eurosym}
\usepackage{graphicx}
\usepackage{color}

\usepackage[hidelinks]{hyperref}
\usepackage[capitalise,noabbrev]{cleveref} % Version 0.18.10


\begin{document}
\mainmatter


\author{Julian M. Kunkel, Jay Lofstead, Colin McMurtrie
}

\title{HPC I/O in the Data Center Workshop (HPC-IODC)}

\institute{
	Deutsches Klimarechenzentrum\\ 	Bundesstraße 45a\\	20146 Hamburg, Germany\\ \email{kunkel@dkrz.de}
	\and
	Center for Computing Research\\ Sandia National Laboratories\\ Albuquerque, USA
	\and
	Swiss National Computing Center (CSCS) \\ Lugano, Switzerland
}

\maketitle{}

% -----------------------------------------------------------------------
\section{Introduction}
\label{sec:intro}

Many public and privately funded data centers host supercomputers for running large scale simulations and analyzing experimental and observational data.
These supercomputers run usually tightly coupled parallel applications that require hardware components that deliver the best performance.
In contrast, commercial data centers, such as Facebook and Google, execute loosely coupled workloads with a broad assumption of regular failures.
The dimension of the data centers is enormous.
A 2013 article summarizes commercial data centers' dimensions\,\cite{data13}.
It estimates, for example, that Facebook hosts around 100\,PB of storage and Google and Microsoft manage around 1 million servers each -- although the hardware is split among several physical data centers -- a modus operandi not suitable for HPC centers.
With the hunger for information, the globally installed storage capacity increases exponentially and is expected to hit 7,235 Exabytes by 2017\,\cite{EXA13}.
This trend is visible in the sales reports of companies such as the disk drive manufacturer Seagate. Within 5 years, they shipped 1 billion HDDs, which means 700.000 units every day\,\cite{SG14}.
With state-of-the-art 8\,TB disks, this would already account for 5.5 exabyte of capacity by day.

Management of the huge amount of data is vital for effective use of the contained information. However, with limited budgets, it is a daunting task for data center operators,
especially as design and storage system required hardware depends heavily on the executed workloads.
A co-factor of the increasing difficulty is the increase in complexity of the storage hierarchy with the adoption of SSD and memory class storage technology.
The US Department of Energy recognizes the importance of data management, listing it among the top 10 research challenges for Exascale\,\cite{top14}. % the document says: “Affordability of data management, both procurement and operational, is a huge challenge.“

There are several initiatives, consortia and special tracks in conferences that target RD\&E audiences.
Examples are the Storage Networking Industry Association (SNIA) for enterprises, the
Big Data and Extreme-Scale Computing (BDEC) initiative\footnote{\url{http://www.exascale.org/bdec/}}, the
Exascale10 workgroup\,\cite{brinkmann14}, the Parallel Data Storage Workshop (PDSW) and the HEC FSIO workshop \cite{bancroft2009}.

There are many I/O workloads studies and performance analysis reports for parallel I/O available.
Additionally, many surveys of enterprise technology usage include predictions of analysis for future storage technology and the storage market such as\,\cite{idc1}.
However, analysis conducted for HPC typically focuses on applications and not on the data center perspective.
Information about data center operational aspects is usually described in file system specific user groups and meetings or described partially in research papers as part of the evaluation environment.

In this workshop, we bring together I/O experts from data centers and application workflows to share current practices for scientific workflows, issues and obstacles for both hardware and the software stack, and R\&D to overcome these issues.


\section{Organization of the Workshop}

\noindent The workshop content was built on three tracks:
\begin{itemize}
  \item \textbf{Research paper presentations} -- authors needed to submit a paper regarding relevant research for I/O in the datacenter.
  \item \textbf{Talks from I/O experts} -- authors needed to submit a rough outline for the talk related to the operational aspects of the data center.
  \item \textbf{Invited track} for a keynote and two moderated discussion slots.
\end{itemize}

\noindent The CFP has been issued beginning of January.
Important deadlines were:
\begin{itemize}
  \item Submission deadline: 2017-04-12 AoE
  \item Author notification: 2017-04-25
  \item Workshop: 2017-06-22
  \item Camera-ready papers: 2017-07-22
\end{itemize}

From all submissions, the programm committee selected four talks from I/O experts and four research papers for presentation during the workshop.

\subsection{Programm Committee}
\begin{itemize}
  \item Wolfgang Frings (\textit{Jülich Supercomputing Center, Germany})
  \item Javier Garcia Blas (\textit{University Carlos III of Madrid, Spain})
  \item  Rob Ross (\textit{Argonne National Laboratory, USA})
  \item   Carlos Maltzahn (\textit{University of California, Santa Cruz, USA})
  \item  Thomas Boenisch (\textit{HLRS, Stuttgart, Germany})
  \item  Sai Narasimhamurthy (\textit{Seagate, United Kingdom})
  \item Jean-Thomas Acquaviva (\textit{DDN, France})
  \item  Julian Kunkel (\textit{DKRZ, Germany})
  \item  Jay Lofstead (\textit{Sandia National Laboratory, USA})
  \item  Colin McMurtrie (\textit{CSCS, Switzerland})
\end{itemize}



\section{Workshop Summary}
\label{sec:summary}

Throughout the day, on average 65 participants attended the workshop.
We had a good mix of talks from I/O experts, data center relevant research followed by a short discussion.
A short summary of the presentations is given in the following.
The slides of the presentations are available on the workshop's webpage: \\
\url{http://wr.informatik.uni-hamburg.de/events/2017/iodc}.



\subsection{Research Papers}

The research session covered 5 accepted papers from 6 submissions:
\begin{itemize}
\item In the first talk, Walker Haddock presented results of the efficiency for GPU offloaded erasure coding for Ceph.
With a GPU plugin to support coding, a 1 GB/s full duplex performance is achievable for 100 shards.
\item Eugen Betke introduced an online monitoring system for parallel I/O performance based on SIOX.
The novelty of the approach is non-intrusive monitoring via an instrumented FUSE mountpoint allowing to cover \texttt{mmap()} operations.
\item Jakob Lüttgau presented a simulator for hierarchical storage systems focusing on tape systems.
Queuing systems are used to model I/O on the different storage tiers;
the simulation allows to measure derived metrics in a fine-grained fashion for instance to analyzing waiting times (quality of service) and drive utilization.
\item Jay Lofstead showed results for a large scale performance study for the IOR benchmark to identify performance variability and stragglers across the different OSTs.
This demonstrated that for each measurement a small proportion of storage targets ($<$ 20\%) are slower than the others but the performance of storage targets chances over time changing the assignment slow/fast targets.
\item The last talk of the research papers by Pilar Gomez-Sanchez introduced a framework to recover the access pattern of MPI parallel applications on a high-level.
The methodology characterizes I/O behavior for individual application phases and introduces several derived metrics.
A process starts another phase when it invokes an instrumented MPI function.
\end{itemize}

\subsection{Talks from Experts}

The seven talks from experts included information about the site and typical
application profiles but also contained information regarding I/O tools and strategies applied to mitigate pressing issues.

\begin{itemize}
\item
In the first talk, Bryan Lawrence introduced the computation infrastructure in the UK with a focus on Earth-Science.
He described the JASMIN infrastructure in detail which is managed and designed by STFC.
A main distinction is that the infrastructure is continuously upgraded and the system architecture is developed by themself and not any vendor.

\item Tiago Quintino shed light on the I/O challenges of ECMF.
The center observes the need to adapt workflows to deal with paradigm shifts in technology.
For example, to move from compute centric to a data centric paradigm minimizing data movement and shipping compute to data.

\item Roland Laifer introduces the HPC systems at KIT.
Highlights of his talk are results on using Lustre on a wide area Infiniband connection, their analysis approach by
capturing I/O statistics, and the approach for disaster recovery.

\item Yuichi Tsujita presented the K Computer and storage systems.
He focused on obstacles due to large scale parallelism: the metadata server load, client eviction (loss of server connections), and cross node interference.
Several strategies to mitigate these issues are presented.

\item Sandra Mendez introduces the LRZ HPC systems and then focuses on monitoring of I/O patterns.
The usage of deployed analysis tools PerSyst and Darshan are illustrated on several applications.

\item Clemens Grelck introduced the cHiPSet project which is an community effort that fosters collaboration in the area of HPC and Big Data.
The activity is still open and new researcher may join the project:
\url{http://www.chipset-cost.eu}

\item Rosemary Francis presented the tool Mistral that profiles I/O of large scale applications to identify bad I/O patterns, foster optimizing and load balancing.
The monitoring tool allows to detect rogue applications based on policies such as limiting storage capacity or metadata.
A case study shows the effectiveness to tame non well-formed I/O.


\end{itemize}

From the individual talks, it can be concluded that analyzing and understanding I/O behavior is the top priority for researchers and data centers.


\subsection{Discussion round}

The major distinguishing feature for this workshop compared to other venues is the discussion rounds.
The opportunity for themed, open discussions about issues both pressing and relevant to the data center
community facilitates sharing experiences, solutions, and problems.

This year we focused on the community development of an IO-500 benchmark, see \url{http://io500.org}.
The IO-500 benchmark consists of data and metadata benchmarks to identify performance boundaries for optimized and suboptimal applications.
Each benchmark is run in an easy and hard mode to identify best-case performance for optimized applications and typical performance for applications with a suboptimal access pattern.

The Virtual Institute for I/O (VI4IO)~\footnote{\url{http://vi4io.org}} supports this activity and tracks comprehensive data from sites, supercomputers and storage on the high-performance storage list.
This data allows for an in-depth analysis of system characteristics and fosters the understanding of I/O systems.

\bibliographystyle{splncs}
\bibliography{literature}


\end{document}
