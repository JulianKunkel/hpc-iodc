\documentclass[a4paper,10pt]{article}
\usepackage[utf8]{inputenc}
\usepackage{hyperref}

\title{HPC I/O in the Data Center}
\author{Julian Kunkel \and Jay Lofstead \and Jean-Thomas Acquaviva}

\makeatletter
\renewcommand{\@seccntformat}[1]{}
\makeatother

\setlength{\parindent}{0cm}
\setlength{\parskip}{0.5em}

\usepackage{url}

\begin{document}

\maketitle


\section{Abstract}
Managing scientific data at a large scale is challenging for scientists but also for the host data center.
The storage and file systems deployed within a data center are expected to meet users' requirements for data integrity and high performance across heterogeneous and concurrently running applications.

With new storage technologies and layers in the memory hierarchy, the picture is becoming murkier.
To effectively manage the data load within a data center, I/O experts must understand how users expect to use these new storage technologies and what services they should provide in order to enhance user productivity. We seek to ensure a systems-level perspective is included in these discussions.

The HPC-IODC workshop is a forum to present and discuss work that addresses the storage challenge from a unique perspective, moving the focus from the application-centric perspective to the perspective of data centers and operators.
In the workshop, we bring together I/O experts from data centers and application workflows to share current practices for scientific workflows, issues and obstacles for both hardware and the software stack, and R\&D to overcome these issues.
To focus on relevant aspects and streamline the discussion, a list of relevant topics is provided as the common structure of the talks.
Trending topics such as machine learning and I/O for AI applications are covered as well.

The IODC workshop covers three tracks:
The research track accepts papers covering state-of-the-practice and research dedicated to storage in the data centre.
The expert talks track revolves around presentations such as a description of the operational aspects of a data centre and 
particular solutions for specific data centre workloads in production.
Lastly, in the student mentoring sessions, the students will be given 10 minutes to talk about what they are working on followed by 10-15 minutes of conversation with the community to help the students progress in their studies.

For further information regarding participation and the topics, see the web page.

\section{Keywords}
Data center, File systems, Storage, Performance, Architecture

\section{Workshop length}
Full day.

\section{Workshop webpage}
\url{https://hps.vi4io.org/events/2026/iodc}


\section{Targeted audience}
\begin{itemize}
\item I/O experts from data centers and industry.
\item Researchers/Engineers working on high-performance I/O for data centers.
\item Interested domain scientists and computer scientists interested in discussing I/O issues.
\item Vendors are also welcome, but their presentations must align with the same topics and not focus on commercial aspects.
\end{itemize}




\section{Estimated attendance}
40-50


\section{Workshop format}
The morning session offers a keynote, invited talks and community activities.
The afternoon session of HPC-IODC puts an emphasis on data center issues and covers presentations regarding state of the practice in the data center, discussion slots, and talks from I/O experts.

Researchers and I/O experts can also submit their proposals for a paper or talk according to a call for participation for papers and speakers.
We provide a common list of topics to be addressed in each I/O expert's talk on our webpage, such that the individual presentations are aligned.

\section{Previous Instances}

This workshop is held since the first time ISC extended it's program to include workshops (ISC 2015).
In 2015, the workshop was held with approximately 30 attendees, 10 invited presentations, and two discussion sessions.
In 2016, the workshop format was changed, in addition to the expert talks, we added a research section allowing submission of short scientific papers and a keynote talk.
The attendance to the workshop was similar to 2015, but finally we accepted four research papers.
In 2017, more than 50 people attended the workshop; we accepted 5 papers. 
Despite the large attendance of previous instances, we were forced by the organizers to run on a half day.
That year we teamed up with WOPSSS (a workshop focusing on performance aspects in I/O) and organized a full storage together promoting the two events on a joint webpage: \url{https://isc-hpc-io.org}.
In 2018, we organized a shared morning session with WOPSSS (45+ people) and then split after lunch into two fully autonomous workshops.
From 2019 onward, WOPSSS moved to a different conference venue and, thus, HPC-IODC was again the only storage-centric workshop at ISC.
Attendance for 2019, 2020, 2022, 2023, 2024, 2025 was between 40 and 50 people in the morning session, with a drop to around 30 in the afternoon session.
The number of research papers on these instances was just a handful but we always organized a program around expert talks and student research talks.
In 2025, the workshop was accepted only without publications, hence we apply in 2026 also in the same track.
However, the similar number of participants and typical success was achieved.

\section{Expected outcome}
1) Sparking impulses for the development of data center I/O.

2) The networking among the participants will be improved.

To increase the participation, we firstly will actively invite representatives of different data centers and representatives of active workflow users/developers to give a talk.
Secondly, we will announce the workshop on typical mailing lists and send the call for participation.




\end{document}
