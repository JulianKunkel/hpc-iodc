\documentclass[a4paper,10pt]{article}
\usepackage[utf8]{inputenc}
\usepackage{hyperref}

\title{VI4IO: The I/O Community Hosting the High-Performance Storage List}
\author{Julian Kunkel}

\makeatletter
\renewcommand{\@seccntformat}[1]{}
\makeatother

\setlength{\parindent}{0cm}
\setlength{\parskip}{0.5em}

\usepackage{url}

\begin{document}

\maketitle

\section{BoF Organizers and Speakers}
\begin{itemize}
  \item Julian Kunkel (DKRZ, Germany), kunkel@dkrz.de
  \item Jay Lofstead (Sandia National Lab, USA), gflofst@sandia.gov
  \item Colin McMurtrie (CSCS, Switzerland), cmurtrie@cscs.ch
\end{itemize}


\section{Short Abstract}  % 250 words maximum
Due to the increasing complexity of data management in high-performance computing, the activities in the storage research community 
increased over the last few years.
In this BoF, the Virtual Institute for I/O (VI4IO), an international effort for the high-performance storage community is introduced and discussed with the participants.
The community web page \url{http://vi4io.org} will be launched at ISC-HPC.

The goals of the VI4IO community are:
\begin{itemize}
  \item To provide a platform for I/O researchers and enthusiasts for exchanging information.
  \item To foster international collaboration in the field of high-performance I/O.
  \item To track deployment of large storage systems by hosting of an open high-performance storage list.
\end{itemize}

In particular, it offers research groups and vendors the chance to describe their knowledge.
Furthermore, relevant I/O tools for, e.g., monitoring and benchmarking are described.
All this information is characterized in meaningful categories and are visualized in tag clouds to simplify search.
The High-Performance Storage List is introduced, similar to TOP500, this list maintains relevant storage systems and tracks their performance.
Since I/O benchmarking is composed of many facets and usually very time consuming, tracked metrics are not handled as strict as for related lists.
However, every site submitting a result is required to clearly describe the measurement conditions.

This open community is guided by philosophical cornerstones:
\begin{itemize}
  \item To treat every member and participant equally.
  \item To allow participation of everybody without a membership fee.
  \item To be an independent organization, thus independent of vendors and research facilities.
\end{itemize}
Thus, everybody will be invited to contribute to the community hub that is maintained as open wiki.

In the BoF we give a short introductory presentation and then focus on introducing and discussing the community hub and the metrics of the HPSL list.
We hope that this effort will contribute to accelerate I/O research and foster collaborations.


\section{Keywords} 
Community building, parallel I/O, data center, storage performance, I/O tools


\section{Targeted audience}
\begin{itemize}
  \item I/O experts from data centers and industry.
  \item Researchers/Engineers working on high-performance I/O for data centers.
  \item Interested domain scientists and computer scientists interested in discussing I/O issues.
\end{itemize}


\section{Estimated number of attendees}
40

\section{Short CVs of the organizers}

\subsection{Julian Kunkel}

Dr. Julian Kunkel is post-doc in the group Scientific Computing at the DKRZ.
Since 2006, Julian has been working on tracing environments and tools for client and server-side I/O.
%In 2013, he defended his thesis about the monitoring and simulation of parallel programs on application and system level. 
Julian is responsible for the University of Hamburg's contributions to several funded projects.  
He is focusing on system-wide monitoring and optimization of parallel I/O.

\subsection{Jay Lofstead}
Dr. Jay Lofstead is a Principal Member of Technical Staff at Sandia National
Laboratories in Albuquerque, New Mexico. Since 2010, Jay has been working on
HPC simulation workflows focusing on data management issues and as well as
general I/O and storage issues for HPC.  His prior work includes the R\&D100
Award winning ADIOS I/O componentization framework in use in more than 30
production scientific simulations. He is a member of several conference and
workshop program committees.

\subsection{Colin McMurtrie}
Colin McMurtrie is an Associate Director at CSCS, in Switzerland, where he is in charge of the Systems Integration group.  At CSCS since 2009, he managed the National Systems group for 4 years before moving into his current role.  Colin has experience with the management of large HPC systems in a production environment and has organised some targeted workshops relating to storage on HPC systems.
\end{document}
