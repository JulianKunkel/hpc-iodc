\documentclass[a4paper,10pt]{article}
\usepackage[utf8]{inputenc}
\usepackage{hyperref}

\title{VI4IO and the IO 500}
\author{Julian Kunkel}

\makeatletter
\renewcommand{\@seccntformat}[1]{}
\makeatother

\setlength{\parindent}{0cm}
\setlength{\parskip}{0.5em}

\usepackage{url}

\begin{document}

\maketitle

\section{BoF Organizers and Speakers}
\begin{itemize}
  \item Julian Kunkel (DKRZ, Germany), kunkel@dkrz.de
  \item Jay Lofstead (Sandia National Lab, USA), gflofst@sandia.gov
  \item Colin McMurtrie (CSCS, Switzerland), cmurtrie@cscs.ch
  \item John Bent (Seagate Government Solutions), john.bent@seagategov.com 
\end{itemize}


\section{Short Abstract}  % 250 words maximum
Due to the increasing complexity of HPC data management, activities in the storage research community 
have increased over the last few years.  The general purpose of this BoF is to foster this community and discuss 
the role of the international Virtual Institute for I/O (VI4IO, \url{http://vi4io.org}) in supporting, developing, and maintaining this community.  
A specific purpose of the BoF is
to discuss whether the community would be well-served by an IO-500 benchmark similar to the Top-500. 


\if 0
In particular, 
VI4IO offers research groups and vendors the chance to describe their knowledge.
Furthermore, relevant I/O tools for, e.g., monitoring and benchmarking are described.
All this information is characterized in meaningful categories and are visualized in tag clouds to simplify search.
\fi

VI4IO is committed to hosting a portal for storage benchmark results including
forums in which researchers can share insights into benchmarking practices
and share related IO tools such as for monitoring.  The exact form of these
storage benchmarks is currently undecided.  One approach will be to allow the
arbitrary submission of any benchmark result along with descriptive measurement
conditions hopefully sufficient to enable reproducibility.  A more controversial approach is to
create an IO-500 modeled after the Top-500.  
We will present a straw-man proposal of an IO-500 to initiate a discussion about the best approach.

VI4IO is guided by philosophical cornerstones and has the following goals:
\begin{itemize}
  \item Treat every member and participant equally
  \item Allow free participation without any membership fees inclusive to all
  \item Be independent of vendors and research facilities
  \item Provide information exchanges for I/O researchers and enthusiasts 
  \item Foster international collaboration in the field of high-performance I/O
  \item Track, and encourage, the deployment of large storage systems by hosting storage benchmark results 
\end{itemize}

Through this BoF, we hope to accelerate I/O research and foster collaborations.

\section{Keywords} 
Community building, parallel I/O, data center, storage performance, I/O tools, benchmarks


\section{Targeted audience}
\begin{itemize}
  \item I/O experts from data centers and industry.
  \item Researchers/Engineers working on high-performance I/O for data centers.
  \item Interested domain scientists and computer scientists interested in discussing I/O issues.
\end{itemize}


\section{Estimated number of attendees}
40

\section{Short CVs of the organizers}

\subsection{Julian Kunkel}

Dr. Julian Kunkel is post-doc in the group Scientific Computing at the DKRZ.
Since 2006, Julian has been working on tracing environments and tools for client and server-side I/O.
%In 2013, he defended his thesis about the monitoring and simulation of parallel programs on application and system level. 
Julian is responsible for the University of Hamburg's contributions to several funded projects.  
He is focusing on system-wide monitoring and optimization of parallel I/O.

\subsection{Jay Lofstead}
Dr. Jay Lofstead is a Principal Member of Technical Staff at Sandia National
Laboratories in Albuquerque, New Mexico. Since 2010, Jay has been working on
HPC simulation workflows focusing on data management issues and as well as
general I/O and storage issues for HPC.  His prior work includes the R\&D100
Award winning ADIOS I/O componentization framework in use in more than 30
production scientific simulations. He is a member of several conference and
workshop program committees.

\subsection{Colin McMurtrie}
Colin McMurtrie is an Associate Director at CSCS, in Switzerland, where he is in charge of the Systems Integration group.  At CSCS since 2009, he managed the National Systems group for 4 years before moving into his current role.  Colin has experience with the management of large HPC systems in a production environment and has organised some targeted workshops relating to storage on HPC systems.

\subsection{John Bent}
As Chief Architect for Seagate Government Solutions, John Bent researches and designs storage systems necessary to support an exascale supercomputer.  His prior work
includes building several burst buffer prototypes while working for EMC and the PLFS virtual file system that achieved orders of magnitude bandwidth improvements for parallel writes to a single file while working for Los Alamos National Lab.

\end{document}
